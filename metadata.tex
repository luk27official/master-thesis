% Type of your thesis:
%	"bc" for Bachelor's
%	"mgr" for Master's
%	"phd" for PhD
%	"rig" for rigorosum
\def\ThesisType{mgr}

% Language of your study programme:
%	"cs" for Czech
%	"en" for English
\def\StudyLanguage{cs}

% Thesis title in English (exactly as in the official assignment)
\def\ThesisTitle{Prediction and visualization of cryptic binding regions}

% Author of the thesis (you)
\def\ThesisAuthor{Lukáš Polák}

% Year when the thesis is submitted
\def\YearSubmitted{2025}

% Name of the department or institute, where the work was officially assigned
% (according to the Organizational Structure of MFF UK in English,
% see https://www.mff.cuni.cz/en/faculty/organizational-structure,
% or a full name of a department outside MFF)
\def\Department{Department of Software Engineering}

% Is it a department (katedra), or an institute (ústav)?
\def\DeptType{Department}

% Thesis supervisor: name, surname and titles
\def\Supervisor{doc. RNDr. David Hoksza, Ph.D.}

% Supervisor's department (again according to Organizational structure of MFF)
\def\SupervisorsDepartment{Department of Software Engineering}

% Study programme (does not apply to rigorosum theses)
\def\StudyProgramme{Computer Science - Software and Data Engineering}

% An optional dedication: you can thank whomever you wish (your supervisor,
% consultant, who provided you with tea and pizza, etc.)
\def\Dedication{%
I would like to thank my supervisor, doc. RNDr. David Hoksza, Ph.D., for his guidance and help in the field of bioinformatics and during the writing of this thesis. I am also grateful to Mgr. Vít Škrhák for his valuable assistance with CryptoBench and the development of the smoothing model. My thanks extend to all colleagues from the Charles University Structural Bioinformatics Group for their insightful feedback and discussions that contributed to the improvement of my work. Above all, I am really thankful to my family and friends, whose support and encouragement have been essential throughout my studies.
}

% AI disclaimer
\def\AIinformation{%
This thesis was written with the assistance of artificial intelligence tools, including but not limited to large language models for text rephrasing and code development. All AI-generated content has been manually reviewed, verified, and integrated by the author. The author takes full responsibility for the accuracy and integrity of all work presented herein.
}

% Abstract (recommended length around 80-200 words; this is not a copy of your thesis assignment!)
\def\Abstract{%
This thesis addresses the prediction and visualization of cryptic binding sites (CBS) in protein structures, which are important for drug discovery. Building on the CryptoBench dataset and prediction model developed earlier at FMP CUNI, we introduce a methodology for clustering residue-level predictions into contiguous cryptic binding sites. The approach is integrated into CryptoShow, a newly created web application that enables users to submit protein structures, obtain CBS predictions, and visualize results interactively. The system supports both public and custom protein structures, leveraging machine learning and modern visualization tools to highlight binding regions and to show potential conformational changes. The thesis details the development of the clustering algorithm, evaluation of the methodology, and the software architecture of CryptoShow.
}

% 3 to 5 keywords (recommended) separated by \sep
% Keywords are useful for indexing and searching for the theses by topic.
\def\ThesisKeywords{%
bioinformatics\sep protein\sep binding sites\sep machine learning
}

% If any of your metadata strings contains TeX macros, you need to provide
% a plain-text version for use in XMP metadata embedded in the output PDF file.
% If you are not sure, check the generated thesis.xmpdata file.
\def\ThesisAuthorXMP{\ThesisAuthor}
\def\ThesisTitleXMP{\ThesisTitle}
\def\ThesisKeywordsXMP{\ThesisKeywords}
\def\AbstractXMP{\Abstract}

% If your abstracts are long and do not fit in the infopage, you can make the
% fonts a bit smaller by this setting. (Also, you should try to compress your abstract more.)
\def\InfoPageFont{}
%\def\InfoPageFont{\small}  % uncomment to decrease font size

% If you are studing in a Czech programme, you also need to provide metadata in Czech:
% (in English programmes, this is not used anywhere)

\def\ThesisTitleCS{Predikce a vizualizace kryptických vazebných míst}
\def\DepartmentCS{Katedra softwarového inženýrství}
\def\DeptTypeCS{Katedra}
\def\SupervisorsDepartmentCS{Katedra softwarového inženýrství}
\def\StudyProgrammeCS{Informatika - Softwarové a datové inženýrství}

\def\ThesisKeywordsCS{%
bioinformatika\sep protein\sep vazebná místa\sep strojové učení
}

\def\AbstractCS{%
Tato diplomová práce se zabývá predikcí a vizualizací kryptických vazebných míst (CBS) v proteinových strukturách, která jsou důležitá při vývoji léčiv. Na základě datasetu CryptoBench a jeho modelu predikce CBS vyvinutého dříve na MFF UK představujeme metodologii pro seskupování predikcí na úrovni aminokyselin do ucelených kryptických vazebných míst. Tento přístup je integrován do nově vytvořené webové aplikace CryptoShow, která umožňuje uživatelům nahrát proteinové struktury, získávat predikce CBS a interaktivně vizualizovat výsledky. Aplikace podporuje jak veřejně dostupné, tak vlastní proteinové struktury a využívá strojové učení a moderní vizualizační nástroje k zvýraznění vazebných oblastí a zobrazení potenciálních konformačních změn mezi strukturami. Práce podrobně popisuje vývoj clusterovacího algoritmu, evaluaci metodologie a softwarovou architekturu aplikace CryptoShow.
}

\chapter*{Introduction}
\addcontentsline{toc}{chapter}{Introduction}

Proteins are essential to a wide range of biological processes. Proteins consist of amino acids, which are the building blocks of these macromolecules. The amino acids form a chain that determines the protein's structure and function. Certain amino acids, referred to as binding, are more likely to interact with other molecules called ligands. This is a fundamental property to drug discovery and research \cite{trainor2007importance}, \cite{ballante2021protein}, \cite{mannhold2006protein}.

Detection of such residues, which are individual amino acids within the protein structure, is a challenging task, and various computational methods have been developed to predict potential binding residues, which are further grouped into distinct binding regions (also referred to as binding sites) based on the location of the residues \cite{krivak2018p2rank}, \cite{le2009fpocket}, \cite{aggarwal2021deeppocket}, \cite{smith2024graph}. Certain binding residues may experience structural changes due to the dynamic nature of these regions, becoming cryptic binding residues (CBRs) that pose greater identification challenges. While current methods may detect cryptic binding sites (CBSs), their models are generally trained to identify any binding residues, rather than being specifically focused on CBSs.

A dataset and methodology for detecting CBRs was previously developed at the Faculty of Mathematics and Physics, Charles University, known as CryptoBench \cite{vskrhak2025cryptobench}. This thesis has two primary objectives. 

The first goal is to extend this approach by clustering individually predicted CBRs from the CryptoBench model into structurally contiguous regions, referred to as CBSs. This methodology may also be applicable to other prediction methods.

The second objective is to develop a user-friendly client-server web application, CryptoShow, which would enable users to submit protein structures and obtain CBR predictions. These predictions would be clustered into distinct binding sites and visualized for the user, including animations that illustrate potential conformational changes between the protein's initial state and analogous structures identified using the AHoJ tool \cite{feidakis2022ahoj}.

This thesis is structured in four chapters, and it is organized as follows:

Chapter 1 provides an overview of the field of bioinformatics, introducing key terminology, concepts, and relevant tools that underpin the subsequent work.

Chapter 2 details the methodological framework, beginning with the prediction of cryptic binding residues using the CryptoBench model. This is followed by a description of the clustering approach for grouping CBRs into cryptic binding sites, and concludes with an explanation of the animation pipeline for visualizing conformational transitions between protein structures.

Chapter 3 presents the design and implementation of the CryptoShow web application. This includes a discussion of the technology stack, architectural and design choices, and an in-depth look at the frontend, particularly the integration of Mol* for 3D structure visualization \cite{sehnal2021mol}. Deployment and maintenance considerations are also addressed.

Finally, Chapter 4 serves as user documentation, outlining the main functionalities of the application and illustrating typical use cases.

\chapter*{Conclusion}
\addcontentsline{toc}{chapter}{Conclusion}

Cryptic binding sites play a crucial role in understanding protein function and facilitating drug discovery. In this thesis, a methodology was introduced for clustering residue-level predictions into coherent cryptic binding sites, incorporating a novel smoothing step using a dedicated machine learning model. The approach was evaluated on the CryptoBench dataset, which comprises protein structures with annotated cryptic binding sites. Results indicate that integrating the clustering technique with the smoothing model enhances performance compared to the original method.

As a next step, the CryptoShow web application was created to offer an intuitive platform for detecting and visualizing cryptic binding sites in protein structures, leveraging the CryptoBench model and the clustering methodology introduced earlier. The application enables users to conveniently examine prediction outcomes and investigate identified cryptic sites. Additionally, integration with the AHoJ tool was implemented, allowing users to search for similar protein structures and visualize possible conformational changes. During the development of the CryptoShow application, we focused on a modular, service-oriented architecture, which allows for easy integration of new features and models in the future.

Potential future enhancements include enabling users to select other structure models of the uploaded protein, as opposed to being limited to the first one, integrating additional predictive models next to the CryptoBench model, and providing the capability to predict cryptic binding sites directly from protein sequences. Furthermore, the clustering approach could be refined, as the current methodology tends to generate a larger number of smaller pockets rather than consolidating them into larger ones, though this task is inherently challenging due to the nature of cryptic binding sites.

We believe that the CryptoShow application will be a valuable tool for researchers in structural bioinformatics and computational drug discovery.

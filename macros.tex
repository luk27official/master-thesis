%%% This file contains definitions of various useful macros and environments %%%
%%% Please add more macros here instead of cluttering other files with them. %%%

%%% Switches based on thesis type

\def\TypeBc{bc}
\def\TypeMgr{mgr}
\def\TypePhD{phd}
\def\TypeRig{rig}

\ifx\ThesisType\TypeBc
\def\ThesisTypeName{bachelor}
\def\ThesisTypeTitle{BACHELOR THESIS}
\fi

\ifx\ThesisType\TypeMgr
\def\ThesisTypeName{master}
\def\ThesisTypeTitle{MASTER THESIS}
\fi

\ifx\ThesisType\TypePhD
\def\ThesisTypeName{doctoral}
\def\ThesisTypeTitle{DOCTORAL THESIS}
\fi

\ifx\ThesisType\TypeRig
\def\ThesisTypeName{rigorosum}
\def\ThesisTypeTitle{RIGOROSUM THESIS}
\fi

\ifx\ThesisTypeName\undefined
\PackageError{thesis}{Unknown thesis type.}{Please check the definition of ThesisType in metadata.tex.}
\fi

%%% Switches based on study program language

\def\LangCS{cs}
\def\LangEN{en}

\ifx\StudyLanguage\LangCS
\else\ifx\StudyLanguage\LangEN
\else\PackageError{thesis}{Unknown study language.}{Please check the definition of StudyLanguage in metadata.tex.}
\fi\fi

%%% Minor tweaks of style

% These macros employ a little dirty trick to convince LaTeX to typeset
% chapter headings sanely, without lots of empty space above them.
% Feel free to ignore.
\makeatletter
\def\@makechapterhead#1{
  {\parindent \z@ \raggedright \normalfont
   \Huge\bfseries \thechapter\quad #1
   \par\nobreak
   \vskip 20\p@
}}
\def\@makeschapterhead#1{
  {\parindent \z@ \raggedright \normalfont
   \Huge\bfseries #1
   \par\nobreak
   \vskip 20\p@
}}
\makeatother

% This macro defines a chapter, which is not numbered, but is included
% in the table of contents.
\def\chapwithtoc#1{
\chapter*{#1}
\addcontentsline{toc}{chapter}{#1}
}

% Draw black "slugs" whenever a line overflows, so that we can spot it easily.
\overfullrule=1mm

%%% Macros for definitions, theorems, claims, examples, ... (requires amsthm package)

\theoremstyle{plain}
\newtheorem{thm}{Theorem}
\newtheorem{lemma}[thm]{Lemma}
\newtheorem{claim}[thm]{Claim}
\newtheorem{defn}{Definition}

\theoremstyle{remark}
\newtheorem*{cor}{Corollary}
\newtheorem*{rem}{Remark}
\newtheorem*{example}{Example}

%%% Style of captions of floating objects (figures etc.)

\ifcsname DeclareCaptionStyle\endcsname
\DeclareCaptionStyle{thesis}{style=base,font=small,labelfont=bf,labelsep=quad}
\captionsetup{style=thesis}
\captionsetup[algorithm]{style=thesis,singlelinecheck=off}
\captionsetup[listing]{style=thesis,singlelinecheck=off}
\fi

%%% An environment for typesetting of program code and input/output
%%% of programs.
\DefineVerbatimEnvironment{code}{Verbatim}{fontsize=\small, frame=single}

% Settings for lstlisting -- program listing with syntax highlighting
\ifcsname lstset\endcsname
\lstset{
  language=C++,
  tabsize=2,
  showstringspaces=false,
  basicstyle=\footnotesize\ttfamily\color{black},
  identifierstyle=\bfseries\color{black},
  commentstyle=\color{green!50!black},
  stringstyle=\color{red!50!black},
  keywordstyle=\color{blue!75!black},
  numberstyle=\color{black},
  columns=fixed,
  basewidth=0.5em,
  fontadjust=true}

% Floating listings, used in the same way as the figure environment
\ifcsname DeclareNewFloatType\endcsname
\DeclareNewFloatType{listing}{}
\floatsetup[listing]{style=ruled}
\floatname{listing}{Program}
\fi

%%% The field of all real and natural numbers
\newcommand{\R}{\mathbb{R}}
\newcommand{\N}{\mathbb{N}}

%%% Useful operators for statistics and probability
\DeclareMathOperator{\pr}{\textsf{P}}
\DeclareMathOperator{\E}{\textsf{E}}
\DeclareMathOperator{\var}{\textrm{var}}
\DeclareMathOperator{\sd}{\textrm{sd}}

%%% Transposition of a vector/matrix
\newcommand{\T}[1]{#1^\top}

%%% Asymptotic "O"
\def\O{\mathcal{O}}

%%% Various math goodies
\newcommand{\goto}{\rightarrow}
\newcommand{\gotop}{\stackrel{P}{\longrightarrow}}
\newcommand{\maon}[1]{o(n^{#1})}
\newcommand{\abs}[1]{\left|{#1}\right|}
\newcommand{\dint}{\int_0^\tau\!\!\int_0^\tau}
\newcommand{\isqr}[1]{\frac{1}{\sqrt{#1}}}

%%% TODO items: remove before submitting :)
\newcommand{\xxx}[1]{\textcolor{red!}{#1}}

%%% Detailed settings of bibliography

\ifx\citet\undefined\else

% Maximum number of authors of a single work. If exceeded, "et al." is used.
%\ExecuteBibliographyOptions{maxnames=2}
% The same setting specific to citations using \citet{...}
\ExecuteBibliographyOptions{maxcitenames=2}
% The same settings specific to the list of literature
%\ExecuteBibliographyOptions{maxbibnames=2}

% Shortening first names of authors: "E. A. Poe" instead of "Edgar Allan Poe"
%\ExecuteBibliographyOptions{giveninits}
% The same without dots ("EA Poe")
%\ExecuteBibliographyOptions{terseinits}

% If your bibliography entries are hard to break into lines, try this mode:
%\ExecuteBibliographyOptions{block=ragged}

% Possibly reverse the names of the authors with the non-ISO styles:
%\DeclareNameAlias{default}{family-given}

% Use caps-and-small-caps for family names in ISO 690 style.
\let\familynameformat=\textsc

% We want to separate multiple authors in citations by commas
% (while we use semicolons in the bibliography as per the ISO standard)
\DeclareDelimFormat[textcite]{multinamedelim}{\addcomma\space}
\DeclareDelimFormat[textcite]{finalnamedelim}{\space and~}

\fi
